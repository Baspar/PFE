\chapter{Problématiques}
    Intéressons aux problématiques successives que devait résoudre le projet, et le programme.\\

    On peut diviser en trois phases différentes la construction de la solution globale:
    \begin{enumerate}
        \item \textbf{Proof-of-Concept} (Élaboration méthode pour un seul utilisateur)
        \item \textbf{Multi-utilisateur} (Gestion plusieurs parcours utilisateur, réalisation de groupe)
        \item \textbf{Stockage} (Gestion optimale de la mise en mémoire morte des données)
    \end{enumerate}.\\

    Pour chacune de ces problématiques, nous allons détailler les différentes motivations pour cette phase, ainsi que les techniques employées pour les résoudre.

    \section{Proof-of-Concept}
        \subsection{Motivations}
            Cette première phase a pour but de mettre en place un système nous permettant une réalisation simplifié du but du projet.\\
            Pour se simplifier la tâche, nous allons nous fixer quelque limitations pour cette phase:
            \begin{itemize}
                \item Gestion d'\textbf{un seul utilisateur}
            \end{itemize}.\\

            Avant de commencer à raisonner sur cette phase, posons quelques définitions:
            \begin{description}
                \item[Document:] Objet représenté par une \underline{adresse unique}
                \item[Session:] Séquence ordonnée de \textit{documents}, représentant le cheminement d'un utilisateur parmi des documents
            \end{description}
            
            Que va-t-il nous rester comme buts?
            \begin{itemize}
                \item Retenir les sessions
                \item Prédire des documents futurs à partir de ces sessions
            \end{itemize}.\\
        \subsection{Résolution}
            L'aspect principal de ce problème va passer par une utilisation des \textbf{chaînes de Markov}.\\
            \begin{center}
                Séquence $X_1, X_2, \ldots , X_n$ de variables aléatoires à valeurs dans l’espace des états possédant la propriété de Markov.
            \end{center}.\\

            Les processus de Markov nous permettent de modéliser les probabilité de passage d'un état à un autre, sachant que cela ne dépend que des n derniers états\\
            Grâce à cela, nous pouvons modéliser les déplacement entre les documents. Sachant qu'il existe différent ordre de Markov, 


    \section{Multi-Utilisateur}
        \subsection{Motivations}
        \subsection{Résolution}
    \section{Stockage}
        \subsection{Motivations}
        \subsection{Résolution}
