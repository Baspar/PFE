\chapter{Problématiques}
    Intéressons aux problématiques successives que devait résoudre le projet, et le programme.\\

    On peut diviser en trois phases différentes la construction de la solution globale:
    \begin{enumerate}
        \item \textbf{Proof-of-Concept} (Élaboration méthode pour un seul utilisateur)
        \item \textbf{Multi-utilisateur} (Gestion plusieurs parcours utilisateur, réalisation de groupe)
        \item \textbf{Stockage} (Gestion optimale de la mise en mémoire morte des données)
    \end{enumerate}.\\

    Pour chacune de ces problématiques, nous allons détailler les différentes motivations pour cette phase, ainsi que les techniques employées pour les résoudre.

    \section{Proof-of-Concept}
        \subsection{Motivations}
            Cette première phase a pour but de mettre en place un système nous permettant une réalisation simplifié du but du projet.\\
            Pour se simplifier la tâche, nous allons nous fixer quelque limitations pour cette phase:
            \begin{itemize}
                \item Gestion d'\textbf{un seul utilisateur}
            \end{itemize}.\\

            Avant de commencer à raisonner sur cette phase, posons quelques définitions:
            \begin{description}
                \item[Document:] Objet représenté par une \underline{adresse unique}
                \item[Session:] Séquence ordonnée de \textit{documents}, représentant le cheminement d'un utilisateur parmi des documents
            \end{description}
            
            Que va-t-il nous rester comme buts?
            \begin{itemize}
                \item Retenir les sessions
                \item Prédire des documents futurs à partir de ces sessions
            \end{itemize}.\\
        \subsection{Résolution}
            L'aspect principal de ce problème va passer par une utilisation des \textbf{chaînes de Markov}.\\
            \begin{center}
                Séquence $X_1, X_2, \ldots , X_n$ de variables aléatoires à valeurs dans l’espace des états possédant la propriété de Markov.
            \end{center}.\\

            Les processus de Markov nous permettent de modéliser les probabilité de passage d'un état à un autre, sachant que cela ne dépend que des n derniers états\\
            Grâce à cela, nous pouvons modéliser les déplacement entre les documents. Sachant qu'il existe différent ordre de Markov qui correspondront chacune au nombre de documents dont dépendra l'état suivant.\\

            \begin{figure}[h]
               \centering
               \includegraphics[scale=0.5]{ressources/graph/Markov1.png}
               \caption{Markov d'ordre 1}
            \end{figure}
            \begin{figure}[h]
               \centering
               \includegraphics[scale=0.5]{ressources/graph/Markov3.png}
               \caption{Markov d'ordre 3}
            \end{figure}

            Les schémas ci-dessus sont une représentation graphique de deux chaines de Markov, les Graphes de chaine de Markov.\\
            En construisant de tels systèmes, il devient aisé en connaissant les $n$ éléments par lesquels nous sommes passé, et en utilisant une Markov d'ordre $n$, de prédire quels seraient les documents les plus pertinent.\\

            En pratique, nous savons comment prédire à partir d'une Markov donnée, et nous pouvons aisément construire des Markovs d'un ordre donnée avec une ou des sessions.\\
            Cependant, cela amène à une question:
            \begin{center}
                Quelles chaine(s) de Markov considérer?
            \end{center}.\\

            Il y a deux inconnues à cette question:
            \begin{itemize}
                \item Combien?
                \item Lesquelles?
            \end{itemize}.\\

            \large{Lesquelles?}\\
            \normalsize
            Du fait que nous possédons une multitude de Markovs (Pour des session de longueur N, on peut construire une markov d'ordre N-1 maximum), combien en utiliser?\\
            Il est clair qu'en utiliser qu'une seule ne serait pas pertinent. Pourquoi le document à suivre ne dépendrait que des 2 documents précedents? Ou des 3?\\
            Il faut donc considérer \textbf{les markovs d'ordre 1 jusqu'à un certain ordre: All-kth-Markov}.

            \large{Combien?}
            \normalsize
            La question qui se pose maintenant est jusqu'à quel ordre?\\
            Pour se faire, reposons nous sur un domaine qui utilise déjà fortement les chaines de Markov: \textbf{le Natural Language Processing} (Traitement du langage naturel).\\
            Dans ce dernier, les chaines de Markov sont utilisé pour prédire les mots suivant en se basant sur un ensemble de phrase donnée en je de test.

            \begin{figure}[h]
               \centering
               \includegraphics[scale=0.5]{ressources/graph/NLP.png}
               \caption{Markov d'ordre 1 chargée avec les phrases "Je mange des légumes" et "Je conduis des voitures"}
            \end{figure}.\\

            Dans ce domaine, ou les chaines de Markov d'ordre $n$ sont apellées $n$-grammes, il est communément admis que considérer des $n$-grammes d'ordre supérieurs a 4 n'apportait rien.\\
            Nous allons donc nous fixer le même chiffre, tout en se gardant une assez grande flexibilité dans le code pour modifier ce chiffre facilement.\\

            Ceci vient conclure cette première version, où tout les objectifs sont atteints.
    \section{Multi-Utilisateur}
        \subsection{Motivations}
          Cette version vient prendre le relai sur la premier.\\
          Son but est simple: réutiliser l'idée qui à été amenée avec la version 1, et la rendre adaptée pour plusieurs utilisateurs.\\

          La manière la plus simple de rendre le principe optimisé pour le traitement de plusieurs utilisateurs serait de \textbf{grouper les utilisateurs}.

          Cela demande cependant plusieurs phases de réflexion pour répondre notamment aux questions suivantes:
          \begin{itemize}
            \item De quelle manière grouper les utilisateur?
            \item Quels utilisateurs doivent être regroupés?
          \end{itemize}
        \subsection{Résolution}
    \section{Stockage}
        \subsection{Motivations}
        \subsection{Résolution}
