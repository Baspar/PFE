\documentclass{beamer}
\usepackage[utf8]{inputenc}
\usepackage{color}
\usepackage{fancyvrb}
\usetheme{Berlin}
%\usepackage{ressource}
\setbeamerfont{caption}{size=\footnotesize}

\title{Représentation probabiliste d’un parcours de navigation dans un corpus documentaire}
\subtitle{Soutenance PFE}
\author{LAINE Bastien}
\institute{Génie Mathématique | INSA Rouen}

\begin{document}
    \beamertemplatenavigationsymbolsempty

    \begin{frame}
        \titlepage{}
    \end{frame}

    \begin{frame}
        \frametitle{Sommaire}
        \tableofcontents[hidesubsections]
    \end{frame}


    \section{Introduction}
        \begin{frame}
            \frametitle{Sommaire}
            \tableofcontents[currentsection, hideothersubsections]
        \end{frame}
        \subsection{}
            \begin{frame}
                \frametitle{Présentation du projet}
            \end{frame}

    \section{État de l'art}
        \begin{frame}
            \frametitle{Sommaire}
            \tableofcontents[currentsection, hideothersubsections]
        \end{frame}
        \subsection{}
            \begin{frame}
                \frametitle{Différentes catégories de systèmes de recommandation}
                Les systèmes de recommandation peuvent être classés dans deux catégories distinctes:
                \pause
                \begin{itemize}
                    \item Approche basée sur le contenu
                    \pause
                    \item Approche basée sur les utilisateurs
                \end{itemize}
            \end{frame}
            \begin{frame}
                \frametitle{Approche basée sur le contenu}
                    Utilisation du profil des documents.
                    pause[2]
                    \only<2-4>{
                        \begin{block}{Approche classique}
                            \begin{itemize}
                                \pause[3]
                                \item Profilage utilisateur
                                \pause[4]
                                \item Recommandation
                            \end{itemize}
                        \end{block}
                    }
                    \pause[5]
                    \only<5>{
                        \begin{exampleblock}{Exemple: films}
                            \begin{center}
                                \begin{tabular}{|c|c|c|c|c|c|c|c|c|c|}
                                    \hline
                                    &Horreur&Épouvante&Action&Romance&Comédie\\
                                    \hline
                                    A&1& &1& &1\\
                                    \hline
                                    B&1&1&1& & \\
                                    \hline
                                    C& & &1& &1\\
                                    \hline
                                    D& & & &1& \\
                                    \hline
                                    E& & &1& & \\
                                    \hline
                                \end{tabular}
                            \end{center}
                        \end{exampleblock}
                    }
            \end{frame}
            \begin{frame}
                \frametitle{Filtrage collaboratif}
                    Utilisation des interactions utilisateur.
                    \pause[2]
                    \only<2-5>{
                        \begin{block}{Approche classique}
                            \begin{itemize}
                                \pause[3]
                                \item Normalisation
                                \pause[4]
                                \item Similitudes utilisateurs
                                \pause[5]
                                \item Recommandation
                            \end{itemize}
                        \end{block}
                    }
                    \pause[6]
                    \only<6>{
                        \begin{exampleblock}{Exemple: films}
                            \begin{center}
                                \begin{tabular}{|c|c|c|c|c|c|c|c|c|c|}
                                    \hline
                                    &HP1&HP2&SW1&SW2&TW\\
                                    \hline
                                    A&4&5& & &1\\
                                    \hline
                                    B&3&4&5&5&2\\
                                    \hline
                                    C& & &2&2&5\\
                                    \hline
                                    D& &3& &4& \\
                                    \hline
                                \end{tabular}
                            \end{center}
                        \end{exampleblock}
                    }
            \end{frame}

    \section{Versions}
        \subsection{v1 - POC}
            \begin{frame}
                \frametitle{Sommaire}
                \tableofcontents[currentsection, currentsubsection]
            \end{frame}
            \begin{frame}
            \frametitle{Problématiques}
            Version "Proof Of Concept"
            \begin{itemize}
            \pause
            \item Comment modéliser les parcours d'un utilisateur?
            \pause
            \item Comment prédire les documents suivants?
            \end{itemize}
            \end{frame}
            \begin{frame}
            \frametitle{Chaîne de Markov}
            Que sont les chaînes de Markov?
            \pause
            \begin{block}{Description}
            Graphe orienté dont les arêtes porte la probabilité de passage d'un nœud (=Ensemble d'états) à un autre.
            \end{block}
            \pause
            \begin{block}{Les chaînes Markov appliquées aux documents}
            Dans notre cas:
            \begin{itemize}
            \item Les \textbf{séquences de documents} représentent les \textbf{nœuds}.
            \item Les \textbf{probabilités de séquence suivante} représentent les \textbf{arêtes}.
            \end{itemize}
            \end{block}
            \end{frame}
            \begin{frame}
            \frametitle{Exemples de chaîne de Markov}
            \pause
            \begin{exampleblock}{Chaîne de Markov d'ordre 1}
            \begin{center}
            \includegraphics[scale=0.5]{graph/Markov1.png}
            \end{center}
            \end{exampleblock}
            \pause
            \begin{exampleblock}{Chaîne de Markov d'ordre 3}
            \begin{center}
            \includegraphics[scale=0.3]{graph/Markov3.png}
            \end{center}
            \end{exampleblock}
            \end{frame}
            \begin{frame}
            \frametitle{Quelle(s) chaîne(s) considérer?}
            Quelles sont les limites?
            \begin{itemize}
            \pause[2]
            \item Combien de chaînes de Markov? \pause[3] $\Rightarrow$ \textbf{All-k}$^{th}$\textbf{-Markov}
            \pause[4]
            \item Ordre de chaîne de Markov maximal? \pause[6] $\Rightarrow$ \textbf{Ordre 4}
            \pause[5]
            \end{itemize}
            \begin{exampleblock}{NLP}
            \begin{center}
            \includegraphics[scale=0.35]{graph/NLP.png}
            \end{center}
            \end{exampleblock}
            \end{frame}
            \begin{frame}
            \frametitle{Implémentation}
            Spécificités:
            \begin{itemize}
            \pause
            \item Langage: \pause C++
            \pause
            \item Stockage: \pause Manuel
            \pause
            \item Utilisation: \pause Invite de commandes/Script
            \end{itemize}
            \end{frame}
            \begin{frame}
            \frametitle{Implémentation}
            \begin{block}{Diagramme de classes}
            \begin{center}
            \only<1>{\includegraphics[scale=0.2]{graph/v1class.png}}
            \pause[2]
            \only<2>{\includegraphics[scale=0.5]{graph/v1classneat.png}}
            \end{center}
            \end{block}
            \end{frame}
            \begin{frame}
            \frametitle{Démonstration}
            \begin{center}
            Démonstration
            \end{center}
            \end{frame}
        \subsection{v2 - Multi-Utilisateur}
            \begin{frame}
                \frametitle{Sommaire}
                \tableofcontents[currentsection, currentsubsection]
            \end{frame}
            \begin{frame}
            \frametitle{Problématiques}
            Version Multi-Utilisateur
            \pause
            \begin{itemize}
            \item Comment grouper les parcours de plusieurs utilisateurs?
            \pause
            \item Comment prédire les documents suivants à partir des parcours de plusieurs utilisateurs?
            \pause
            \item Comment grouper efficacement plusieurs utilisateurs?
            \pause
            \item Comment rendre l'application plus "interfacable"?
            \end{itemize}
            \end{frame}
            \begin{frame}
            \frametitle{Union de Markov}
            Comment joindre plusieurs Markov de même ordre?
            \pause[2]
            \begin{columns}[t]
            \begin{column}{3cm}
            \begin{exampleblock}{User 1}
            \begin{center}
            \includegraphics[scale=0.25]{graph/mergeUser1.png}
            \end{center}
            \end{exampleblock}
            \end{column}
            \begin{column}{3cm}
            \begin{exampleblock}{User 2}
            \begin{center}
            \includegraphics[scale=0.25]{graph/mergeUser2.png}
            \end{center}
            \end{exampleblock}
            \end{column}
            \pause[3]
            \begin{column}{3cm}
            \begin{exampleblock}{Union}
            \begin{center}
            \only<3>{\includegraphics[scale=0.25]{graph/mergeUsers.png}}
            \pause[4]
            \only<4>{\includegraphics[scale=0.25]{graph/mergeUsersFix.png}}
            \end{center}
            \end{exampleblock}
            \end{column}
            \end{columns}
            \end{frame}
            \begin{frame}
            \frametitle{Union de Markov}
            Une autre possibilité gardant les probabilités.\\
            \pause[5]\textbf{Plus clair, mais moins efficace}
            \pause[2]
            \begin{columns}[t]
            \begin{column}{3cm}
            \begin{exampleblock}{User 1}
            \begin{center}
            \includegraphics[scale=0.25]{graph/wrongMergeUser1.png}
            \end{center}
            \end{exampleblock}
            \end{column}
            \begin{column}{3cm}
            \begin{exampleblock}{User 2}
            \begin{center}
            \includegraphics[scale=0.25]{graph/wrongMergeUser2.png}
            \end{center}
            \end{exampleblock}
            \end{column}
            \pause[3]
            \begin{column}{3cm}
            \begin{exampleblock}{Union}
            \begin{center}
            \only<3>{\includegraphics[scale=0.25]{graph/wrongMergeUsers.png}}
            \pause[4]
            \only<4,5>{\includegraphics[scale=0.25]{graph/wrongMergeUsersFix.png}}
            \end{center}
            \end{exampleblock}
            \end{column}
            \end{columns}
            \end{frame}
            \begin{frame}
            \frametitle{Groupement d'utilisateur}
            Quel est le but du groupement d'utilisateurs?
            \pause
            \begin{itemize}
            \item Créer des groupes d'utilisateurs partagent les mêmes intérêts
            \pause
            \item Réunir les utilisateurs \textbf{proches} en terme d'intérêt
            \pause
            \item Définir une position "d'intérêt utilisateur"
            \end{itemize}
            \end{frame}
            \begin{frame}
            \frametitle{Groupement d'utilisateur}
            Comment obtenir une distance/position d'intérêt?
            \pause
            \begin{columns}[t]
            \begin{column}{4cm}
            \begin{block}{Vecteur session moyen}
            \pause
            Avantages:
            \begin{itemize}
            \pause
            \item Récupérable via sessions
            \end{itemize}
            \pause
            Inconvénients:
            \begin{itemize}
            \pause
            \item Stockage sessions
            \pause
            \item Taille sessions variable
            \pause
            \item Complexité
            \end{itemize}
            \end{block}
            \end{column}
            \pause
            \begin{column}{4cm}
            \begin{block}{Vecteur catégorie moyen}
            \pause
            Avantages:
            \begin{itemize}
            \pause
            \item Facile à calculer
            \pause
            \item Pas de stockage de sessions
            \pause
            \item Taille fixe
            \end{itemize}
            \pause
            Inconvénients:
            \begin{itemize}
            \pause
            \item Nécessité de catégories
            \end{itemize}
            \end{block}
            \end{column}
            \end{columns}
            \end{frame}
            \begin{frame}
            \frametitle{Groupement d'utilisateur}
            \only<-4,6-7>{Une fois les positions obtenues, comment grouper les utilisateurs?}
            \pause[2]
            \only<-4,6-7>{
            \begin{block}{Techniques de partitionnement}
            \begin{itemize}
            \pause[3]
            \item Classification ascendante hiérarchique
            \pause[4]
            \item Classification descendante hiérarchique
            \pause[5]
            \item Maximum de vraisemblance
            \pause[7]
            \item K-Moyennes (K-Means)
            \pause[8]
            \end{itemize}
            \end{block}
            }
            \only<5>{
            \begin{exampleblock}{Classification ascendante/descendante hiérarchique}
            \begin{center}
            \includegraphics[scale=0.3]{graph/hierachiqueOrder.png}
            \end{center}
            \pause[6]
            \end{exampleblock}
            }
            \only<8>{
            \begin{exampleblock}{K-Means}
            \begin{figure}[h]
            \centering
            \includegraphics[scale=0.3]{images/kmeans.png}
            \caption{Source: codeproject.com}
            \end{figure}
            \end{exampleblock}
            }
            \end{frame}
            \begin{frame}
            \frametitle{Implémentation}
            \only<-7>{
            Spécificités:
            \begin{itemize}
            \pause[2]
            \item Langage: \pause[3] Java
            \pause[4]
            \item Stockage: \pause[5] Annulé
            \pause[6]
            \item Utilisation: \pause[7] "Librairie"
            \end{itemize}
            \pause[8]
            }
            \only<8>{
            \begin{block}{Diagramme de classes}
            \begin{center}
            \includegraphics[scale=0.1]{graph/v2class.png}
            \end{center}
            \end{block}
            \pause[9]
            }
            \only<9>{
            \begin{block}{Diagramme de classes}
            \begin{center}
            \includegraphics[scale=0.2]{graph/v2classneat.png}
            \end{center}
            \end{block}
            }
            \end{frame}
        \subsection{v3 - Stockage}
            \begin{frame}
                \frametitle{Sommaire}
                \tableofcontents[currentsection, currentsubsection]
            \end{frame}
            \begin{frame}
            \frametitle{Problématiques}
            Version Stockage
            \pause
            \begin{itemize}
            \item Comment stocker efficacement l'ensemble des données du programme?
            \pause
            \item Comment récupérer des données sans avoir à tout charger en mémoire?
            \pause
            \item Comment visualiser de manière plus intuitive les données?
            \end{itemize}
            \end{frame}
            \begin{frame}
            \frametitle{Différentes solutions de stockage}
            Deux grandes solutions:
            \pause
            \begin{columns}[t]
            \begin{column}{4cm}
            \begin{block}{SGBD}
            \pause
            Avantages
            \begin{itemize}
            \pause
            \item Vitesse d'accès
            \pause
            \item Sécurité
            \end{itemize}
            \pause
            Inconvénients
            \begin{itemize}
            \pause
            \item Complexité de prise en main
            \pause
            \item Nouvelle phase de conception
            \end{itemize}
            \end{block}
            \end{column}
            \pause
            \begin{column}{4cm}
            \begin{block}{Fichier}
            \pause
            Avantages
            \begin{itemize}
            \pause
            \item Peu de modifications
            \pause
            \item Facile d'accès
            \end{itemize}
            \pause
            Inconvénients
            \begin{itemize}
            \pause
            \item Chargement intégral des objets
            \pause
            \item "Bidouillage"
            \end{itemize}
            \end{block}
            \end{column}
            \end{columns}
            \end{frame}
            \begin{frame}
            \frametitle{Choix du SGBD}
            Deux types de SGBD:
            \pause
            \begin{itemize}
            \item SQL (MySQL, PostgreSQL, ...)
            \pause
            \item NOSQL (Cassandra, MongoDB, ...)
            \end{itemize}
            \pause
            \begin{block}{NOSQL (Not Only SQL)}
            \pause
            \begin{itemize}
            \item Clef/Valeur (Riak)
            \pause
            \item Orienté colonne (Cassandra)
            \pause
            \item Orienté document (MongoDB)
            \pause
            \item \textbf{Orienté graphe (Neo4j)}
            \end{itemize}
            \end{block}
            \end{frame}
            \begin{frame}
            \frametitle{Neo4j}
            \only<-5,7>{Qu'est-ce que Neo4j?}
            \pause[2]
            \only<-5,7>{
            \begin{block}{Description}
            SGBD NOSQL open-source Java orienté graphe développé depuis 2000.
            \end{block}
            }
            \pause[3]
            \only<-5,7>{
            \begin{block}{Caractéristiques}
            \pause[4]
            \begin{itemize}
            \item Nombreux connecteurs (Java, Ruby, R, ...)
            \pause[5]
            \item Langage synthétique (Cypher)
            \pause[6]
            \item Interface Web incluse
            \pause[8]
            \end{itemize}
            \end{block}
            }
            \only<6>{
            \begin{exampleblock}{Cypher}
            \begin{center}
            \includegraphics[scale=0.2]{images/cypher.png}
            \end{center}
            \end{exampleblock}
            \pause[7]
            }
            \only<8>{
            \begin{exampleblock}{Interface Web}
            \begin{center}
            \includegraphics[scale=0.1]{images/neo4jbrowser.png}
            \end{center}
            \end{exampleblock}
            }
            \end{frame}
            \begin{frame}
            \frametitle{Implémentation}
            Spécificités:
            \begin{itemize}
            \pause
            \item Langage: \pause Java
            \pause
            \item Stockage: \pause Neo4j
            \pause
            \item Utilisation: \pause Invite de commandes/Librairie
            \pause
            \item Petits ajouts:
            \pause
            \begin{itemize}
            \item Durée de vie sessions
            \pause
            \item Pourcentage minimum vecteur utilisateur
            \end{itemize}
            \end{itemize}
            \end{frame}
            \begin{frame}
                \frametitle{Modèle Neo4j}
                \only<-1>{
                    \begin{block}{Graphe de base}
                        \begin{center}
                            \includegraphics[scale=0.2]{graph/modeleNeo4j.png}
                        \end{center}
                    \end{block}
                    \pause[2]
                }
                \only<2>{
                    \begin{block}{Diagramme de classes}
                        \begin{center}
                            \includegraphics[scale=0.17]{graph/v3classneat.png}
                        \end{center}
                    \end{block}
                }
            \end{frame}
            \begin{frame}
                \frametitle{Démonstration}
                \begin{center}
                    Démonstration
                \end{center}
            \end{frame}

    \section{Conclusion}
        \subsection{}
            \begin{frame}
                \frametitle{Conclusion}
            \end{frame}
\end{document}
