\documentclass{beamer}
\usepackage[utf8]{inputenc}
\usetheme{Berlin}
%\usepackage{ressource}

\title{Représentation probabiliste d’un parcours de navigation dans un corpus documentaire}
\subtitle{Soutenance PFE}
\author{LAINE Bastien}
\institute{Génie Mathématique | INSA Rouen}

\begin{document}
    \begin{frame}
        \titlepage{}
    \end{frame}

    \section*{Sommaire}

    \begin{frame}
        \tableofcontents[hideallsubsections]
    \end{frame}

    \section{Introduction}
        \begin{frame}
        \end{frame}

    \section{État de l'art}
        \subsection{État de l'art}
            \begin{frame}
            \end{frame}

    \section{Versions}
        \subsection{v1 - POC}
            \begin{frame}
                \frametitle{Problématiques}
                Version "Proof Of Concept"
                \begin{itemize}
                    \pause
                    \item Comment modéliser les parcours d'un utilisateur?
                    \pause
                    \item Comment prédire les documents suivants?
                \end{itemize}
            \end{frame}
            \begin{frame}
                \frametitle{Chaîne de Markov}
                \begin{block}{Description}
                    Graphe orienté dont les arêtes porte la probabilité de passage d'un nœud à un autre.
                \end{block}
                \pause
                \begin{block}{Markov appliqué aux documents}
                    Dans notre cas:
                    \begin{itemize}
                        \item Les \textbf{séquences de documents} représentent les \textbf{nœuds}.
                        \item Les \textbf{probabilités de séquence suivante} représentent les \textbf{arêtes}.
                    \end{itemize}
                \end{block}
            \end{frame}
            \begin{frame}
                \frametitle{Exemples de chaîne de Markov}
                \pause
                \begin{exampleblock}{Markov d'ordre 1}
                    \begin{center}
                        \includegraphics[scale=0.5]{graph/Markov1.png}
                    \end{center}
                \end{exampleblock}
                \pause
                \begin{exampleblock}{Markov d'ordre 3}
                    \begin{center}
                        \includegraphics[scale=0.3]{graph/Markov3.png}
                    \end{center}
                \end{exampleblock}
            \end{frame}
            \begin{frame}
                \frametitle{Quelle(s) chaîne(s) considérer?}
                \begin{itemize}
                    \pause[2]
                    \item Combien de Markov? \pause[3] $\Rightarrow$ \textbf{All-k}$^{th}$\textbf{-Markov}
                    \pause[4]
                    \item Ordre de Markov maximal? \pause[6] $\Rightarrow$ \textbf{Ordre 4}
                    \pause[5]
                \end{itemize}
                \begin{exampleblock}{NLP}
                    \begin{center}
                        \includegraphics[scale=0.35]{graph/NLP.png}
                    \end{center}
                \end{exampleblock}
            \end{frame}
            \begin{frame}
                \frametitle{Implémentation}
                \begin{itemize}
                    \pause
                    \item Langage  \pause $\Rightarrow$ C++
                    \pause
                    \item Stockage \pause $\Rightarrow$ Manuel
                \end{itemize}
            \end{frame}
            \begin{frame}
                \frametitle{Implémentation}
                \pause
                \begin{block}{Diagramme de classes}
                    \begin{center}
                        \includegraphics[scale=0.2]{graph/v1class.png}
                    \end{center}
                \end{block}
            \end{frame}
            \begin{frame}
                \frametitle{Implémentation}
                \begin{block}{Diagramme de classes}
                    \begin{center}
                        \includegraphics[scale=0.5]{graph/v1classneat.png}
                    \end{center}
                \end{block}
            \end{frame}
            \begin{frame}
                \frametitle{Démonstration}
            \end{frame}
        \subsection{v2 Groupement}
        \subsection{v3 Stockage}
\end{document}
